\documentclass[11pt,a4paper]{article}
\newtheorem{assumption}{Assumption}
\usepackage[a4paper, top=2.5cm, bottom=2.5cm, left=3.5cm, right=3.5cm]{geometry}
\usepackage[utf8]{inputenc}
%bibliography
\usepackage[style=authoryear, maxcitenames=2]{biblatex}
\addbibresource{bibliography.bib}
\usepackage{packages}
\linespread{1.5}
\newcommand{\boldsf}[1]{\textbf{\textsf{#1}}}
\begingroup

\author {
        Jørgen Baun Høst\thanks{In loving memory of Norredine Remaoun.}\\
        \small \textit{University of Copenhagen} \\
        \small Department of Economics}


\title{Love thy neighbor?\\
\vspace{0.35cm}
\large An empirical test of neighborhood ethnicity change and Schelling behavior\thanks{I am grateful for the guidance given by my supervisors, Andreas Bjerre-Nielsen and Nikolaj Arpe Harmon, in conjunction with valuable input by both Jeppe Søndergaard Johansen and Jan Moritz Johanning. I also thank the Data Science Lab team at Statistics Denmark for giving me access to the unique datasets used in this paper.}
\vspace{0.35cm}
}
\date{\today}
\setcounter{tocdepth}{2}
\begin{document}
\begin{spacing}{1}
\maketitle    
\end{spacing}

\vspace*{-1cm}
\begin{abstract}
\noindent
Does the ethnicity of your nearest neighbor affect your propensity to move? To provide causal evidence on this question, I use comprehensive administrative data with precise geospatial information and implement a nearest-neighbor research design that compares households within the same neighborhood who receive different-type neighbors as their nearest neighbors to ones who receive them "just down the road". I find asymmetry in residential responses based on neighbor ethnicity: Native Danish households increase their propensity to move within 2 years by approximately 1.6 percent compared to their baseline exit rate, when they receive a new non-Western neighbor. In contrast, non-Western households show no such response. This effect is primarily driven by low-SES native households responding to low-SES non-Western neighbors. These findings provide causal evidence of individually motivated segregation as theorized by \textcite{schelling1971dynamic}, though with more modest effects than documented in the United States.

\medskip
\noindent
\textbf{Keywords:} Schelling, segregation, ethnicity, neighborhood, KNN

\medskip
\noindent
\textbf{JEL classification:} J15, R23 
\end{abstract}


\newpage 
\begingroup
\begin{spacing}{1.25}
\tableofcontents
\end{spacing}
\endgroup


\vspace{1.5cm}
\begin{center}
\textbf{Character Count:} $42,000$ 
\end{center}

\pagebreak
\subfile{sections/1_introduction.tex}

\pagebreak

\subfile{sections/2_model.tex} 

\pagebreak

\subfile{sections/3_data.tex} 

\pagebreak

\subfile{sections/4_results.tex} 

\pagebreak

\subfile{sections/5_conclusion.tex}

\pagebreak

\printbibliography

\pagebreak


\appendix
\appendixpage
%Avoid all sections of TOC be printed
\addtocontents{toc}{\protect\setcounter{tocdepth}{0}}

\subfile{sections/appendix.tex}



\end{document}