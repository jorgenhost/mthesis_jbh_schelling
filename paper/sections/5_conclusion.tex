\documentclass[../main.tex]{subfiles}

\begin{document}
\section{Conclusion}
\label{sec:conclusion}
This paper provides empirical evidence of Schelling behavior among Danish households by examining how the ethnicity of their nearest neighbors affects their propensity to move. Using a nearest neighbor research design by \textcite{Bayer_2022_nearest_neighbor} that compares households within the same neighborhood to those who receive a new different-type neighbors "just down the road", I identify plausibly causal effects of Schelling behavior.

My results show that native Danish households increase their propensity to move within 2 years by approximately 0.3 percentage points, which translate into a 1.6 percent increase relative to their baseline exit rate. This effect is robust across specifications, including controls for household (equivalised) income and wealth in conjunction with tenure and age. In contrast, non-Western show a substantially smaller and statistically insignificant response to receiving native neighbors. 

Importantly, there is significant heterogeneity embedded within these responses. I find that Schelling behavior is primarily concentrated among low-SES native household responding to other low-SES non-Western households. I find an effect of around 2.8 percent increase from their baseline exit rate, nearly twice the magnitude observed in the full sample. This suggests that residential sorting patterns are already pronounced in relatively "poorer" neighborhoods where resources and outside options are likely to be more constrained.

These findings contribute to our understanding of segregation dynamics in several ways. First, they provide causal evidence of Schelling's theoretical prediction that even mild preferences regarding the identity of your (nearest) neighbor can influence the choice of residence. Second, the socioeconomic gradient in responses highlight the intersection of ethnicity and economic resources.

My findings also suggest that the magnitude of Schelling behavior in Denmark is relatively modest compared to the US. Here, \textcite{Bayer_2022_nearest_neighbor} finds symmetric responses in new different-type neighbors among Black and White American households with increases in moving propensities of 6 and 4 percent relative to their baseline exit rates within 2 years, respectively. 

In conclusion, this research demonstrates individually motivated segregation, as theorized by Schelling over five decades ago, remains a relevant mechanism in shaping residential sorting patterns in Denmark today. 
\end{document}