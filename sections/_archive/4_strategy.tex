\documentclass[../main.tex]{subfiles}
\begin{document}

\section{Empirical strategy}
\label{sec:empirical_strategy}
Essentially: Concentration of (past) immigrant neighbour predicts future native "flight". My starting point for determining the propensity to move in response to a new non-native neighbor is the model by \textcite{bramoulle2009_peer_effects} and \textcite{bouncing_with_the_joneses_grenestam}. Consider the following outcome-on-outcome model of social interaction:

\begin{equation}
    y_{it} = \alpha_i + \beta W_i \mathbf{y}_t + \gamma \mathbf{X}_{it} + \delta W_i \mathbf{X}_t + \epsilon_{it},~\mathbb{E}[\epsilon | \mathbf{X}_{it}] = 0
\label{eq:start_endo_soc_effect}
\end{equation}

Where the outcome, binary choice to move or not, of person $i$, denoted $y_{it}$ is a function of a $n x 1$ vector $\mathbf{y}_t$, the contemporaneous choice to move for other households. This is multiplied by $W_i$, the $i^{th}$ row of the $n x n$ interaction matrix $\mathbf{W}$. Each each element $w_{ij}=1$ if household $j$ is a neighbor of person $i$ and zero otherwise.

$\beta$ denotes the endogenous social interaction effect, that is effect of a new neighbor on my propensity to move. This is the focus in this paper. The model from \textcite{bramoulle2009_peer_effects} also includes the exogenous social effect $\delta$ on the propensity to move. In other words, the effect of my neighbors observable background characteristics on my decision to move. 

To elicit meaningful insight into the endogenous social effect, I use plausibly exogenous variation in the number of new non-west neighbors caused by them moving. Consider therefore the following modification of equation (\ref{eq:start_endo_soc_effect}):

\begin{equation}
    y_{it} = \underset{(?)}{\alpha_i} + \beta D_{ik} \mathbf{y}_{t-1} + \gamma \mathbf{X}_{it} + \epsilon_{it},~\mathbb{E}[\epsilon | \mathbf{X}_{it}] = 0
\label{eq:spec1_endo_soc_effect}
\end{equation}

Where the time subscript $t=0, -1, ..., -n$ denotes the time of when the incumbent person decides to move-out.\footnote{For clarity, this happens at $t=0$ for every household such that $y_{i0}=1$.} Each row in $\mathbf{y}_{t-1}$ is given by:

\begin{equation}
    y_{ij} = 
    \begin{cases} 
    1 ~ &\text{if $j$ moves in}  \\
    0 ~ &\text{if $j$ stays} \\
    -1 ~ &\text{if $j$ moves out}
    \end{cases}
\end{equation}


Each element $d_{ij}$ is given by: \footnote{Can be modified to mixed (at least non-native in household), MENAPT, west (excl. DK), etc...} Specifically, each element $d_{ij}$ in $\mathbf{D}_{t-1}$ is given by:

\begin{equation}
    d_{ij} = 
    \begin{cases} 
    1 ~ &\text{if $j$ is a neighbor of $i$ and is of non-native descent}  \\
    0 ~ &\text{otherwise}
    \end{cases}
\end{equation}

I argue that for a meaningful social interaction to manifest itself, we need to allow incumbent persons to react to the arrival of a new (non-west) neighbor. This specification implies that treatment is defined by a change to the neighborhood ethnicity composition. Thus, if a new neighbor does not change the ethnicity composition ($D_{ik} = 0 \rightarrow D_{ik}\mathbf{y}_{t-1}=0$), there is no treatment.




To elicit insight into heterogeneity, use (generalized) causal forest? Look at heterogeneity function of age, years of education? \textbf{SES} 

Look at different otucomes; outcome $y$ prob. of marying differing type(?).

\subsection{Intermission}
\textcite{bramoulle2009_peer_effects}:
The "triad" route(?):

\begin{equation}
    y_i = \alpha + \beta y_{i-1} + \gamma x_{i} + \delta x_ {i-1} + \epsilon_i
\end{equation}

3 students, $i,j,k$. 

Lags of $x_i$ may be used as instruments for $y_{i-1}$. That is some exogenous measure may be an instrument for the outcome of my neighbor. The neighbors (k) of my (i) new neighbor (j) drew him/her to my neighborhood, but if they are not my neighbor, they have no influence on my decision. In other words instrument characteristics (SES?) of k on the decision to move by j. Will have no correlation with the effect on my propensity to move, but will have on my new neighbor.

In the context of my social interaction Schelling model:

Using lagged "SES" as instrument for $y_{i-1}$, the outcome of my neighbor. 

\end{document}
