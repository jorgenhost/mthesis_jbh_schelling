\documentclass[../main.tex]{subfiles}
\begin{document}
\section{Introduction}

Residential segregation, the uneven distribution of ethnic groups within a confined geographic area, remains a persistent feature of Danish society, yet there is limited empirical evidence on how ethnic background directly influences residential sorting. To this end, I seek to answer the fundamental question: To what extent do households respond to the ethnic identity of their nearest neighbors when making residential location decisions?

%Despite this, limited evidence exists of how ethnic neighborhood composition directly affects the decision to move. This likely reflects the complexity involving credible identification of the importance of neighbor identity.

\textcite{schelling1971dynamic} remains the foundational piece in the literature on theoretical determinants of segregation. The key prediction of \textcite{schelling1971dynamic} is that neighborhoods may experience a rapid outflow of residents belonging to a majority, or "tip", when the share of (new) residents belonging to a minority reach a certain threshold. \textcite{schelling1971dynamic} noted that this phenomenon may happen even if majority residents are relatively tolerant of other minorities.\footnote{To illustrate, I have simulated a simple version Schelling's model, which takes inspiration from \textcite{luca_mingarelli}. It can be found in Appendix \ref{sec:appendix_schelling_model_simulation}.} This paper focuses on what Schelling called the "third" kind of segregation - individually motivated segregation - as distinct from organized segregation (such as historical Jim Crow laws) or economically induced segregation (clustering by income or education).  I do not make the claim that organized or economically induced segregation do not exist or play a role in Denmark (perhaps less so than in the United States), but that individual preferences, even if it is not the main driver of observed segregation, still matters and deserves attention. 

Identifying the causal role of preferences for neighborhood composition is vital for understanding persistent segregation. It has been widely established that neighborhoods, both who live there and amenities within, matter for long-run outcomes. For instance, \textcite{chetty2016effects} show the positive impact on earnings and college attendance when presented with the opportunity to move to a more affluent neighborhood. In a Danish context, \textcite{hasager2024sick_poor_neighborhood} show the importance of neighborhood (both at the building and parish level) composition on long-term health outcomes using the quasi-experiment of the Danish Spatial Policy for refugee placement. \textcite{damm2014crime} use the same policy as part of their empirical framework and finds that neighborhood crime rates is strongly associated with youth criminal behavior. In an American context, \textcite{caetano2017school} argue much of the residential sorting found in the US is the result of the parental preferences for the racial and socioeconomic composition of schools, which can lead schools to "tip" with school segregation even greater than neighborhood segregation. \textcolor{red}{CITE THE PAPER BY ANDREAS ON SCHOOL CHOICE} 

Perhaps most similar to me is \textcite{rockwool_boje2024immigrants}, who examine native flight in Denmark, both at the neighborhood and building level. They find that an increase in 30 percentage points in the share of non-Western foreigners in the neighborhood increase the propensity for natives to move out in any given year by around 8 pct. \textcolor{red}{IS THIS THE CORRECT RESULT FROM THEIR PAPER} 

Work by \textcite{schelling1971dynamic} has spawned a rich literature on segregation from which many papers draw their inspiration from.  \textcite{card2008tipping} employ a single "tipping point" Schelling model in an RDD setting. They find discontinuity in the rate of change in minority shares ranging between 5 pct. (Portland, Oregon) to around 20 pct. (Los Angeles, California). Inspired by \textcite{card2008tipping},\textcite{bohlmark_willen_2020_tipping} estimate neighborhood tipping points in Sweden's largest cities (Stockholm, Malmo and Gothenburg) that range between 17-19 pct. minority shares.

\textcite{blair2017outside} further builds on \textcite{card2008tipping} and the idea of neighborhood tipping points in the United States, but points to the importance of outside options in this context. \textcite{blair2017outside} argues  that while White American households may dislike neighbors from a different ethnic background, if the only feasible option is to move to neighborhood not to their likening, they may as well stay. Here, outside option refers to the set of neighboring census tracts that exists within a household's Metropolitan Statistical Area (MSA). 

In this paper, I aim to credibly estimate how households respond directly to receiving a new different-type neighbor.


The paper is structured as follows. 

\subsection{Definitions}
\label{sec:intro_definitions}
In the paper, I define 3 mutually exclusive types of households. (i) \textit{Native} households, where all members are of Danish origin; (ii) \textit{non-Western} households, where at least one member is of non-Western origin and (iii) \textit{Western} households, where at least one household member is non-Native and of Western origin, but with no members that have \textit{non-Western} origin. I follow the definition of Western/non-Western countries by \textcite{west_non_west_def_dst}.\footnote{Throughout the paper, I repeatedly write about "same"- and "different"-type neighbors. To be clear, same-type neighbors refers to two or more households that live in close proximity to each other and both fall within the same category defined above. If they differ in terms of household type, they are different-type neighbors.}

\end{document}







% 1) Determine treat/control
% - Treat: New non-West among K=3
% - Control: New non-west among K=4,..,40? 
% - Drop everyone else
% 2) Make panel. use lf.join_where(pl.date_range(..., eager=True)
% 3) Take unique ID's and collect
% - Inc, edu, emp
% 4) just have a little looksy @ BBR / DAR

%Section \ref{sec:model} outlines and  simulates a simplified of the Schelling model. Section \ref{sec:data} describes the data used in this paper. The empirical strategy is presented in Section \ref{sec:empirical_strategy}. The main results, including a heterogeneity analysis (?), are presented in Section \ref{sec:results}. In Section \ref{sec:Discussion}, I discuss some stuff(?). In Section \ref{sec:conclusion} I summarize and conclude.
Cont...

Include some stats:

\begin{figure}[H]
\centering
\caption{Different-type neighbors}
    \begin{subfigure}{0.47\textwidth}	
	\centering
    \includegraphics[width=\linewidth]{figs/mix_non_west_pos_nn_1985_2020.pdf}
	\caption{Non-Western neighbors} 
	\end{subfigure}	
    \begin{subfigure}{0.47\textwidth}	
	\centering
    \includegraphics[width=\linewidth]{figs/native_nn_1985_2020.pdf}
	\caption{Danish/native neighbors} 
	\end{subfigure}	
      
    \label{fig:avg_k_nearest_dk}
    \begin{minipage}{.9\linewidth}
        \footnotesize \textit{Note}: Non-Western households refers to households where at least one household member is of non-Western origin. Danish/native households refers to households where all members is of Danish descent.
    \end{minipage}
\end{figure}

Some figures yet to be determined where to place can be found in Appendix \ref{sec:appendix_figs}
