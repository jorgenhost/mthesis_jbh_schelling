\documentclass[../main.tex]{subfiles}
\begin{document}
\section{Main results}
\label{sec:results}

This section presents estimates of Schelling behavior among native and non-Western households in Denmark. I find evidence that the propensity to move for native households increases with the arrival of new different-type household among their nearest neighbors rather than slightly further away, consistent with the prediction of the segregation model by \textcite{schelling1971dynamic}. 

Table \ref{tab:main_results_native} and \ref{tab:main_results_non_west} report estimates $\beta_1-\beta_2$ from equation \ref{eq:main_eq_schelling_behavior} as proxy for Schelling behavior. All columns include neighborhood-by-quarter fixed effects ($\omega_{j,t}$ from equation \ref{eq:main_eq_schelling_behavior}) to facilitate comparable analysis. The first column has no further covariates added. 

For native households, the estimate indicate an increase in moving propensity of around 0.3 percentage points if a native households gets a new non-west neighbor among their three nearest neighbors compared to receiving one slightly further away (ranks 4-6). This represents a 1.6 percent relative to the baseline moving rate of 20.23 percent moving rate. Importantly, this effect remains statistically significant and remarkably stable across different specifications. It is worth noting that this includes adding controls for real equivalised household income and/or real net wealth. These are two covariates I would expect to influence an incumbent household's propensity to move.  

In contrast, Table \ref{tab:main_results_non_west} reveals that non-Western households show a substantially smaller response: 0.06-0.1 percentage points or around 0.5 percent relative to the baseline exit rate. This effect is not statistically significant for any specification and suggests that they are unaffected by the identity of a new different-type neighbor. This asymmetry in responses may reflect several factors. First, as \textcite{blair2017outside} notes, outside options matter - if non-Western households perceive limited alternatives in equally desirable neighborhoods, they may be less responsive to receiving native neighbors. Second, the historical context of Danish immigration patterns means that non-Western households may already expect to live in mixed neighborhoods, whereas some native households hold different expectations. 

Comparing these findings to \textcite{Bayer_2022_nearest_neighbor}, who examined racial preferences in the US, reveals interesting cross-country differences. While I find asymmetric responses where only native households exhibit significant responses, \textcite{Bayer_2022_nearest_neighbor} find symmetric responses among both Black and White households, with magnitudes of 6 and 4 percent increases relative to their baseline exit rates within 2 years, respectively. My estimated effect for native households (1.6 percent) is substantially smaller than for either group. This difference may reflect institutional variation in housing market and integration policies, but also the distinct neighborhood contexts in which these "experiments" occur. As I have shown previously, most of these experiments in Denmark tend to happen in high-dense neighborhood, whereas \textcite{Bayer_2022_nearest_neighbor} finds that these "experiments" mostly take place in suburban-sized neighborhoods. These contrasts suggest that while Schelling mechanisms operate across different settings, their magnitude and symmetry are shaped by local spatial, social and policy contexts. 



\begin{table}[H]
    \caption{Estimates of Schelling behavior (native households)}
    \label{tab:main_results_native}
    \begin{adjustbox}{width = \linewidth, center}
        
    \begin{threeparttable}
            \begin{tabular}{lcccc}
\toprule
  & (1) & (2) & (3) & (4) \\ 
\midrule
New diff neighbor $k_{nearest}$ v $k_{near}$ & 0.357*** & 0.368*** & 0.354*** & 0.328*** \\ 
 & (0.084) & (0.084) & (0.081) & (0.081) \\ 
\midrule
N & 5,365,811 & 5,365,811 & 5,365,811 & 5,365,811 \\ 
Neighborhood-by-quarter FE & X & X & X & X \\ 
Mean of dependent variable & 20.23 & 20.23 & 20.23 & 20.23 \\ 
Number of neighborhoods & 3444 & 3444 & 3444 & 3444 \\ 
Income &  & X & X & X \\ 
Tenure &  &  & X & X \\ 
Age &  &  &  & X \\ 
\bottomrule
\end{tabular}
    \begin{tablenotes}[flushleft]
    \item \scriptsize * p < 0.05, ** p < 0.01, *** p < 0.001. Standard errors (in parenthesis) are clustered at the neighborhood level. The table reports the estimate of $\beta_1 - \beta_2$ from equation \ref{eq:main_eq_schelling_behavior}. Table \ref{tab:main_results_full} contains the complete set of coefficients.
    \end{tablenotes}
    \end{threeparttable}
    \end{adjustbox}
\end{table}

\begin{table}[H]
    \caption{Estimates of Schelling behavior (non-Western households)}
    \label{tab:main_results_non_west}
    \begin{adjustbox}{width = \linewidth, center}
    
    \begin{threeparttable}
            \setlength{\LTpost}{0mm}
\begin{longtable}{lcccc}
\toprule
  & (1) & (2) & (3) & (4) \\ 
\midrule
New diff-type neighbor \$k\_\{nearest\}\$ v \$k\_\{close,20\}\$ & 0.774*** & 0.774*** & 0.443** & 0.355** \\ 
 & (0.134) & (0.131) & (0.137) & (0.126) \\ 
N & 1814152 & 1814152 & 1814152 & 1814152 \\ 
Neighborhood-by-quarter FE & X & X & X & X \\ 
Mean of dependent variable & 20.31 & 20.31 & 20.31 & 20.31 \\ 
Number of neighborhoods & 3333 & 3333 & 3333 & 3333 \\ 
Income &  & X & X & X \\ 
Tenure &  &  & X & X \\ 
Age &  &  &  & X \\ 
\bottomrule

\begin{minipage}{\linewidth}
* p < 0.05, ** p < 0.01, *** p < 0.001\\
Standard errors (in parenthesis) are clustered at the municipality-year level.\\
Neighborhoods are heuristically aggregated from Nabolagsatlas.dk to contain a minimum of 500 people.\\
\end{minipage}
    \begin{tablenotes}[flushleft]
    \item \scriptsize * p < 0.05, ** p < 0.01, *** p < 0.001. Standard errors (in parenthesis) are clustered at the neighborhood level. The table reports the estimate of $\beta_1 - \beta_2$ from equation \ref{eq:main_eq_schelling_behavior}. Table \ref{tab:main_results_full_non_west} contains the complete set of coefficients.
    \end{tablenotes}
    \end{threeparttable}
    \end{adjustbox}    
\end{table}

\subsection{Heterogeneity}
The full set of coefficients for tables \ref{tab:main_results_native} and \ref{tab:main_results_non_west}, which can be found in tables \ref{tab:main_results_full} and \ref{tab:main_results_full_non_west}, also indicate noteworthy heterogeneity, particularly along the income and wealth dimension. As mentioned previously, \textcite{blair2017outside} emphasizes that outside options - which are often constrained by economic resources - also affects how households respond to neighborhood demographic changes. This motivates me to elicit insight into the differences in response by socioeconomic status (SES). I define SES as either "\textit{low}" or "\textit{high}": 

\begin{enumerate}
    \item \textbf{Low}: Households earn less than 200,000 DKK in real equivalised terms and was outside the labor marker \textit{or} the best educated individual had less than or equal to 11 years of schooling.
    \item \textbf{High}: Household earn at least 600,000 DKK in real equivalised terms and at least one member had full time work \textit{or} the best educated individual had more than or equal to 18 years of schooling (corresponding to a completed long-cycle/MSc education.
\end{enumerate}

The heterogeneity in response by SES is presented in table \ref{tab:main_results_ses} and \ref{tab:main_results_ses_non_west} for native and non-Western households, respectively. The first two columns in both tables denotes instances where the incumbent households fall within either of these categories. The next 4 columns "pairs" incumbent household by SES with incoming different-type neighbors that also fall within the categories above.

What is immediately clear from these tables is that the Schelling response is primarily driven by low-SES native households "paired" with low-SES non-Western household, who show an increase in moving propensity by around 0.56 percentage points when they receive a new non-Western neighbor or around 2.8 percent increase from their baseline exit rate. This effect is nearly twice the magnitude observed in the full sample. Furthermore, the remaining columns in table \ref{tab:main_results_ses} also exhibit a fascinating pattern. It is incredibly rare for low-SES native households to receive high-SES non-Western and for low-SES non-Western households to receive high-SES native households, which corroborates the idea of the power of residential sorting that happens at the neighborhood level. 
\begin{landscape}
\begin{table}[H]
    \centering
    \caption{Estimates of Schelling behavior (native households) by SES}
    \label{tab:main_results_ses}
    \begin{adjustbox}{width = 0.65\linewidth, center}    
    \begin{threeparttable}
            \begin{tabular}{lcccccc}
\toprule
 & \multicolumn{6}{c}{Move within 2 years (=100)} \\ 
\cmidrule(lr){2-7}
 & SES: Low & SES: High & SES: Low v Low & SES: Low v High & SES: High v High & SES: High v Low \\ 
\cmidrule(lr){2-2} \cmidrule(lr){3-3} \cmidrule(lr){4-4} \cmidrule(lr){5-5} \cmidrule(lr){6-6} \cmidrule(lr){7-7}
  & (1) & (2) & (3) & (4) & (5) & (6) \\ 
\midrule
New diff neighbor $k_{nearest}$ v $k_{near}$ & 0.557*** & 0.287 & 0.558*** & 0.275 & -1.050 & 0.036 \\ 
 & (0.090) & (0.297) & (0.105) & (0.453) & (0.887) & (0.442) \\ 
\midrule
N & 5,883,637 & 609,652 & 3,614,630 & 156,497 & 55,008 & 310,061 \\ 
Neighborhood-by-quarter FE & X & X & X & X & X & X \\ 
Mean of dependent variable & 18.89 & 20.75 & 17.83 & 17.01 & 23.65 & 20.22 \\ 
Number of neighborhoods & 3451 & 3450 & 3451 & 3248 & 2688 & 3446 \\ 
Income &  & X &  &  & X & X \\ 
Tenure & X & X & X & X & X & X \\ 
Age & X & X & X & X & X & X \\ 
\bottomrule
\end{tabular}
    \begin{tablenotes}[flushleft]
    \item \scriptsize * p < 0.05, ** p < 0.01, *** p < 0.001. Standard errors (in parenthesis) are clustered at the neighborhood level. The table reports the estimate of $\beta_1 - \beta_2$ from equation \ref{eq:main_eq_schelling_behavior}. Table \ref{tab:main_results_ses_full} contains the complete set of coefficients.
    \end{tablenotes}
    \end{threeparttable}
    \end{adjustbox}
\end{table}
\begin{table}[H]
    \centering
    \caption{Estimates of Schelling behavior (non-Western households) by SES}
    \label{tab:main_results_ses_non_west}
    \begin{adjustbox}{width = 0.65\linewidth, center}    
    \begin{threeparttable}
            \begin{tabular}{lcccccc}
\toprule
 & \multicolumn{6}{c}{Move within 2 years (=100)} \\ 
\cmidrule(lr){2-7}
 & SES: Low & SES: High & SES: Low v Low & SES: Low v High & SES: High v High & SES: High v Low \\ 
\cmidrule(lr){2-2} \cmidrule(lr){3-3} \cmidrule(lr){4-4} \cmidrule(lr){5-5} \cmidrule(lr){6-6} \cmidrule(lr){7-7}
  & (1) & (2) & (3) & (4) & (5) & (6) \\ 
\midrule
New diff neighbor $k_{nearest}$ v $k_{near}$ & 0.008 & -1.452* & 0.020 & 0.491 & -1.654 & -1.551 \\ 
 & (0.115) & (0.624) & (0.126) & (0.552) & (1.381) & (0.803) \\ 
N & 1,984,581 & 169,445 & 1,609,988 & 127,667 & 37,413 & 117,717 \\ 
Neighborhood-by-quarter FE & X & X & X & X & X & X \\ 
Mean of dependent variable & 18.33 & 25.58 & 18.26 & 18.12 & 28.08 & 26.20 \\ 
Number of neighborhoods & 3447 & 3135 & 3444 & 3361 & 2383 & 3087 \\ 
Income &  & X &  &  & X & X \\ 
Tenure & X & X & X & X & X & X \\ 
Age & X & X & X & X & X & X \\ 
\bottomrule
\end{tabular}
    \begin{tablenotes}[flushleft]
    \item \scriptsize * p < 0.05, ** p < 0.01, *** p < 0.001. Standard errors (in parenthesis) are clustered at the neighborhood level. The table reports the estimate of $\beta_1 - \beta_2$ from equation \ref{eq:main_eq_schelling_behavior}. Table \ref{tab:main_results_ses_non_west_full} contains the complete set of coefficients.
    \end{tablenotes}
    \end{threeparttable}
    \end{adjustbox}
\end{table}
\end{landscape}

The spatial decay of effects shown in tables \ref{tab:main_results_full} and \ref{tab:main_results_full_non_west} provides additional support for the Schelling mechanism. The moving response decreases monotonically with the distance to new different-type neighbors, consistent with a model where the intensity of neighbor interactions diminishes with physical separation.

To assess the robustness of these findings, I follow \textcite{Bayer_2022_nearest_neighbor} and estimate an alternative specification where I combine all control distance into a single category:
\begin{equation}
\begin{split}
    Y_{i, j, t} = \gamma_1 \mathbb{I}[e', k=n_{nearest}] + \gamma_2 \mathbb{I}[e', k = n_{control}] + \gamma Z_{i, j, t} + \omega_{j, t} + \epsilon_{i, j, t}
\label{eq:main_eq_schelling_behavior_I_control}
\end{split}
\end{equation}

Where the coefficient of interest is $\gamma_1-\gamma_2$, the difference in moving propensity following a new different-type households among their three nearest neighbor compared to rank 4 to 40. While this specification increases statistical power, as table \ref{tab:main_results_robust_I_control} shows, my preferred specification distinguishes between neighbor ranks to better capture the spatial decay of effects that better aligns with the theory of \textcite{schelling1971dynamic}.


\begin{table}[H]
    \centering
    \caption{Estimates of Schelling behavior, combined control}
    \label{tab:main_results_robust_I_control}
    \begin{threeparttable}
            \begin{tabular}{lcc}
\toprule
 & Native & Non-Western \\ 
\cmidrule(lr){2-2} \cmidrule(lr){3-3}
 & \multicolumn{2}{c}{Move within 2 years (=100)} \\ 
\cmidrule(lr){2-3}
  & (1) & (2) \\ 
\midrule
New diff neighbor $k_{nearest}$ v $k_{control}$ & 0.981*** & 0.540*** \\ 
 & (0.070) & (0.123) \\ 
 \midrule
N & 5,365,811 & 1,795,109 \\ 
Neighborhood-by-quarter FE & X & X \\ 
Mean of dependent variable & 20.23 & 19.96 \\ 
Number of neighborhoods & 3444 & 3332 \\ 
Income & X & X \\ 
Tenure & X & X \\ 
Age & X & X \\ 
\bottomrule
\end{tabular}
    \begin{tablenotes}[flushleft]
    \item \scriptsize * p < 0.05, ** p < 0.01, *** p < 0.001. Standard errors (in parenthesis) are clustered at the neighborhood level. The table reports the estimate of $\gamma_1 - \gamma_2$ from equation \ref{eq:main_eq_schelling_behavior_I_control}. Table \ref{tab:main_results_robust_I_control_full} contains the complete set of coefficients. 
    \end{tablenotes}
    \end{threeparttable}
\end{table}

The evidence presented here provides empirical support for Schelling's theoretical prediction that even mild preferences regarding the identity of their neighbor - the \textit{third} type of segregation as he called it - can generate segregation. The response of native households, in particular those with low-SES that tend to live in high-dense neighborhoods, indicate that residential sorting based on ethnicity remains an active mechanism in Danish housing markets. 

\end{document}